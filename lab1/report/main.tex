\documentclass[a4paper,14pt]{extarticle}
\input{/home/acesk/Documents/latex/preamble.tex}

\newcommand{\Code}[1]{\textit{#1}}

\begin{document}

\begin{titlepage}
\newgeometry{left=1.18in,top=0.60in,right=0.39in,bottom=0.59in}
\begin{center}
    \uppercase{\textbf{Минобрнауки России\\
            Санкт-Петербургский государственный\\
            электротехнический университет\\
            «ЛЭТИ» им. В.И.Ульянова (Ленина)
    }}
    \vspace{0.25cm}

    \textbf{Кафедра ВТ}
    \vfill

    \uppercase{\textbf{\large{
        Отчет
    }}}
    \\
    \textbf{\large{
      по лабораторной работе №3\\
      по дисциплине «Операционные системы»\\
      Тема: Процессы и потоки\\
      \vspace{0.5cm}
    }}
  \bigskip
\end{center}
\vfill

\begin{tabularx}{\textwidth}{@{}lcXr}
    Студент гр. 8307 & \hspace{1.6cm} & \rule{5cm}{1pt} & Репин С.А.
\end{tabularx}

\vspace{0.5cm}

\noindent
\begin{tabularx}{\textwidth}{@{}lcXr}
    Преподаватель & \hspace{2cm} & \rule{5cm}{1pt} & Тимофеев А.В.
\end{tabularx}

\hfill \break
\hfill \break

\begin{center}
  Санкт-Петербург\\2020
\end{center}

\end{titlepage}



\renewcommand*{\thepage}{}
\tableofcontents
\clearpage
\renewcommand*{\thepage}{\arabic{page}}

\setcounter{page}{3}

\anonsection{Цель работы}

Исследовать управление файловой системой с помощью Win32 API.

\anonsection{Введение}

При выполнении лабораторной работы на языке программирования C стандарта C11
было разработано консольной приложение, управление которым происходит через
различные меню, содержащие подпункты и подменю, которые соответствуют пунктам
заданий. Исходный код приложения доступен на GitHub
\footnote{\url{https://github.com/stnrepin/os_labs/tree/master/lab1}}.

\begin{table}[H]
    \centering
    \caption{Описание файлов в проекте}
    \begin{tabularx}{\textwidth}{|c|X|}
        \hline
        Файл & Описание \\
        \hline
        menu.c & Определение типов и функций для работы с меню \\
        \hline
        main.c & Точка входа в программу; объявления конкретных меню и
                    переходов между ними \\
        \hline
        actions.c & Реализация функций непосредственно выполняющих требования
                    заданий (другими словами, callback'и конечных пунктов меню)
                    \\
        \hline
        error.c & Описание номеров ошибок, а также функции отображения
                    сообщений об ошибках \\
        \hline
    \end{tabularx}
\end{table}

Сборка проекта производится с помощью Powershell-скрипта \Code{build.ps1}
(следует создать папку \Code{build} и запускать скрипт из нее). Также
потребуется пакет Build Tools for Visual Studio 2019.


\section{Управление дисками, каталогами и файлами.}


\subsection{Вывод списка дисков}

Используя функции \Code{GetLogicalDrives} и \Code{GetLogicalDriveStrings},
программа получается набор строк, содержащих названия логических дисков, и
отображает их на экране.

\addtwoimghere{res/11.png}{res/11_real.png}{0.6}{%
Результат выполнения программы и его проверка с помощью Проводника}{}


\subsection{Вывод информации о диске}

Используя функции \Code{GetDriveType}, \Code{GetVolumeInformation},
\Code{GetDiskFreeSpace}, программа получает информацию о произвольном
существующем диске и отображает ее.

\addimghere{res/12.png}{0.45}{Результат выполнения программы}{}


\subsection{Создание и удаление директорий}

Используя функции \Code{CreateDirectory} и \Code{RemoveDirectory}, программа
создает и удаляет директории соответственно.

Заметим, что в программа также добавлена возможность передачи аргумента
командной строки, при получении которого программа меняет текущую директории
при запуске с помощью функции \Code{SetCurrentDirectory}.

\addimghere{res/13.png}{res/13_real.png}{0.6}{%
Результат выполнения программы и его проверка с помощью Проводника}{}

\end{document}

