\documentclass[a4paper,14pt]{extarticle}
\input{/home/acesk/Documents/latex/preamble.tex}

\usepackage{fancyvrb}

\newcommand{\Code}[1]{\textit{#1}}

\setlength{\extrarowheight}{.5ex}

\begin{document}

\begin{titlepage}
\newgeometry{left=1.18in,top=0.60in,right=0.39in,bottom=0.59in}
\begin{center}
    \uppercase{\textbf{Минобрнауки России\\
            Санкт-Петербургский государственный\\
            электротехнический университет\\
            «ЛЭТИ» им. В.И.Ульянова (Ленина)
    }}
    \vspace{0.25cm}

    \textbf{Кафедра ВТ}
    \vfill

    \uppercase{\textbf{\large{
        Отчет
    }}}
    \\
    \textbf{\large{
      по лабораторной работе №3\\
      по дисциплине «Операционные системы»\\
      Тема: Процессы и потоки\\
      \vspace{0.5cm}
    }}
  \bigskip
\end{center}
\vfill

\begin{tabularx}{\textwidth}{@{}lcXr}
    Студент гр. 8307 & \hspace{1.6cm} & \rule{5cm}{1pt} & Репин С.А.
\end{tabularx}

\vspace{0.5cm}

\noindent
\begin{tabularx}{\textwidth}{@{}lcXr}
    Преподаватель & \hspace{2cm} & \rule{5cm}{1pt} & Тимофеев А.В.
\end{tabularx}

\hfill \break
\hfill \break

\begin{center}
  Санкт-Петербург\\2020
\end{center}

\end{titlepage}



\renewcommand*{\thepage}{}
\tableofcontents
\clearpage
\renewcommand*{\thepage}{\arabic{page}}

\setcounter{page}{3}

\anonsection{Цель работы}

Исследовать механизмы управления виртуальной памятью Win32.

\anonsection{Введение}

При выполнении лабораторной работы на языке программирования C стандарта C11
было разработано консольной приложение, управление которым происходит через
меню, пункты которого соответствуют пунктам
заданий. Исходный код приложения доступен на GitHub
\footnote{\url{https://github.com/stnrepin/os_labs/tree/master/lab3}}.

\begin{table}[H]
    \centering
    \begin{tabularx}{\textwidth}{|c|X|}
        \hline
        Файл & Описание \\
        \hline
        menu.c & Определение типов и функций для работы с меню \\
        \hline
        main.c & Точка входа в программу; объявления конкретных меню и
                    переходов между ними \\
        \hline
        actions.c & Реализация функций непосредственно выполняющих требования
                    заданий (другими словами, callback'и конечных пунктов меню)
                    \\
        \hline
        error.c & Описание номеров ошибок, а также функции отображения
                    сообщений об ошибках \\
        \hline
    \end{tabularx}
    \caption*{Таблица 1 Описание файлов в проекте}
\end{table}

Сборка проекта производится с помощью Powershell-скрипта \Code{build.ps1}
(следует создать папку \Code{build} и запускать скрипт из нее). Также
потребуется пакет Build Tools for Visual Studio 2019.

Стоит обратить внимание на некоторые флаги компилятора, с которыми
производилась сборка:
\begin{itemize}
    \item \Code{/O2}. Указывает использовать набор оптимизаций по скорости
    \item \Code{/GL}. Включает оптимизацию всей программы (оптимизация
        производится на основе владения информацией обо всех модулях программы,
        обычно она производится только на основе одного compilation
        unit)
    \item \Code{/Fa}. Создает дополнительно файлы сгенерированного
        ассемблерного кода. Удобно для анализа работы оптимизаций
    \item \Code{/arch:AVX2}. Включает использование расширения x86\_64 AVX2
    \item \Code{/openmp}. Подключает библиотеку OpenMP
\end{itemize}

\clearpage

\section{Реализация многопоточного приложения с использованием функций Win32
API}


Используя функции Win32 API (\Code{CreateThread()}, \Code{SuspendThread()},
\Code{ResumeThread()}, \Code{CreateEvent()}, \Code{SetEvent()},
\Code{WaitForMultipleObjects()}) была написана многопоточная программа, которая
распределяя итерации по потокам блоками в $830716\cdot10$ итераций, вычисляет число
$\pi$ по следующей формуле:
\begin{equation}
    \pi = \frac{1}{N} \sum\limits_{i=0}^{N-1} \frac{4}{1+x_i^2},
    \quad x_i = \frac{i+0.5}{N}, \quad N = 100\;000\;000
\end{equation}
В итоге были получены следующие результаты замеров времени выполнения программы
в зависимости от числа потоков (время измеряется в миллисекундах):
\begin{table}[H]
    \centering
    \begin{tabularx}{\textwidth}{|C|C|C|C|C|C|C|}
        \hline
        № & \multicolumn{6}{c|}{Число потоков} \\
        \cline{2-7}
        & 1 & 2 & 4 & 8 & 12 & 16 \\
        \hline
        1 & 235.265 & 113.653 & 56.288 & 55.988 & 56.618 & 56.234 \\
        \hline
        2 & 233.383 & 114.756 & 56.179 & 56.296 & 56.131 & 56.451 \\
        \hline
        3 & 234.940 & 114.601 & 56.405 & 55.999 & 56.416 & 56.693 \\
        \hline
        \hline
          & 234.529 & 114.337 & 56.291 & 56.094 & 56.388 & 56.459  \\
        \hline
    \end{tabularx}
\end{table}

Построим график зависимости времени выполнения программы от количества потоков
(приведем два графика, чтобы разница по времени при большом количестве потоков
была различима).

\addimghere{res/win1.png}{1}{Зависимость времени от числа потоков (Win32 API)}{}
\addimghere{res/win2.png}{1}{Зависимость времени от числа потоков (Win32 API)}{}

Как видим, максимальное быстродействие достигается при 4 потоках (изменения
времени в дальнейшем можно считать погрешностью относительно изменений времени
до этого).

\section{Реализация многопоточного приложения с использованием технологии
OpenMP}

OpenMP -- это открытый стандарт API для разработки параллельных программ для C,
C++, Fortran. В данной работе он используется как альтернативный способ
выполнения вычислений, описанных в предыдущем разделе. С помощью директив
\Code{parallel}, \Code{for}, \Code{schedule}, \Code{reduction} реализуется
программа по своей работе аналогичная программе, созданной в предыдущем разделе.
Соответственно, ожидается, что время выполнения в обоих случаях будет
идентичным.

Были получены следующие результаты замеров времени выполнения программы
в зависимости от числа потоков (время измеряется в миллисекундах):

\begin{table}[H]
    \centering
    \begin{tabularx}{\textwidth}{|C|C|C|C|C|C|C|}
        \hline
        № & \multicolumn{6}{c|}{Число потоков} \\
        \cline{2-7}
        & 1 & 2 & 4 & 8 & 12 & 16 \\
        \hline
        1 & 232.053 & 114.239 & 56.267 & 55.576 & 56.341 & 56.646 \\
        \hline
        2 & 231.792 & 115.124 & 56.534 & 56.456 & 56.401 & 56.112 \\
        \hline
        3 & 232.797 & 114.441 & 56.319 & 56.096 & 56.130 & 56.416 \\
        \hline
        \hline
          & 232.214 & 114.601 & 56.373 & 56.043 & 56.290 & 56.391 \\
        \hline
    \end{tabularx}
\end{table}

Заметим, что, как и предполагалось выше, эти значения практически не отличаются
от значений, полученных в предыдущем разделе.

Как и ранее, построим графики.

\addimghere{res/omp1.png}{1}{Зависимость времени от числа потоков (OpenMP)}{}
\addimghere{res/omp2.png}{1}{Зависимость времени от числа потоков (OpenMP)}{}

\clearpage

\anonsection{Вывод}

В ходе выполнения данной лабораторной работы были получены знания и
практические навыки разработки многопоточной программы в Windows, используя как
примитивы Win32 API, так и технологию OpenMP. В обоих случаях, время выполнения
программы и точность вычислений не отличались. Экспериментально было получено
оптимальное количество потоков для выполнения данной задачи --- $4$. Связно это
главным образом с тем, что машина, на которой запускалась программа была
четырехядерной (значит, что дальнейшее возрастание числа потоков не может
сказываться на быстродействие из-за того, что выполняют операции в конкретный
момент времени все равно только 4 ядра, а время нахождение потоков в состоянии сна
минимально). Кроме того, было продемонстрировано, как применение потоков
благоприятно сказывается на времени выполнения программы (как минимум в данной
задачи).

При написании приложения с использованием WinAPI была изучена организация
потоков в ОС Windows и API ОС для тесного с ними взаимодействия. Были получены
знания о жизненном цикле потоков (это было необходимо для динамического
распределения работы по потокам), а также был рассмотрен способ реализации
ожидания освобождения потоков с помощью событий.

На этом примере была продемонстрирована простота, которую предоставляет
технология OpenMP по сравнению с WinAPI, позволяя абстрагировать от
низкоуровневых и платформозависимых деталей реализации параллельных программ.
Такое упращение очень важно для написании серьезных и крупных приложений, так
как минимизирует возможность ошибиться при написании кода, учитывая
трудоемкость реализации правильного многопоточного кода и его отладку.

При проектировании и разработке программы нельзя было пройти мимо общих
понятий: состояние гонки, примитивы синхронизации, приоритеты потоков,
написание программ оптимальных с точки зрения доступа к кэшу процессора и
предсказания переходов, знание о которых позволило написать корректное и быстрое
приложение. Во время оптимизации программы пришлось познакомиться с некоторыми
оптимизациями компилятора: встроенные функции (для бесплатных абстракций),
разворачивание цикла (для уменьшения числа итераций и, соответственно,
сравнений в цикле, в котором в основном находятся потоки при выполнении), применение
векторных инструкций (для ускорения самой частой операции --- выполнения i-ой
итерации в формуле вычисления числа $\pi$) и другие.

\end{document}

